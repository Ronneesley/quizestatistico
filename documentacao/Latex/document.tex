\documentclass[12pt]{article}

\usepackage[utf8]{inputenc}
\usepackage[brazil]{babel}
\usepackage{indentfirst}
\usepackage{setspace}
\usepackage[a4paper, left=3cm, right=2cm, top=3cm, bottom=2cm]{geometry}

\setlength{\parindent}{1.5cm}
\setlength{\parskip}{1cm} % espaço entre paragrafos

%\onehalfspacing espaçamento de 1.5
%\doublespacing espaçamento duplo
%\setstretch{3} espaçamento personalizado
%\hspace{2cm} para aumentar o espaço na linha de inicio de so um paragrafo, em qualquer lugar.
%\vspace{2cm} so um paragrafo
%\newline % nova linha
%\rewpage % quebra pagina
%font size
%\begin{*nome-de-ambiente*}
%conteúdo...
%\end{*nome-de-ambiente*} para um paragrafo so spacing para o personalizado


\begin{document}
	\title{\textbf{ INSTITUTO FEDERAL DE EDUCAÇÃO, CIÊNCIA E TECNOLOGIA GOIANO (IF)\\CAMPUS CERES\\  BACHARELADO EM SISTEMAS DE INFORMAÇÃO\\PROGRAMAÇÃO  WEB}} 
	
	\author{Ana Clara Lacerda da Silva\\ Daianny Evillin Costa de Oliveira\\ Elyas Augusto Ferreira de Oliveira\\ Emanuel Gonçalves Menezes\\ Geovana Silva Matuzinho\\ Guilherme Henrique Cândido de Moraes\\ João Fellipe Lemos Costa\\ João Pedro Inacio Porto Vidigal\\ José Pedro Alves Neto\\ Laura Sousa Lima\\ Láyza Ferreira Lopes\\ Leonardo Ferreira Maia\\ Lucas Martins Jesus Oliveira\\ Luiz Gustavo Alves Alencar\\ Maria Eduarda de Sá\\ Mayko Diouzef Mendes do Amaral\\ Ramildo Pereira da Silva Junior\\ Rayllander Antonio Matias de Morais\\ Uigor Teodoro Martins\\ Wagner Martins Rocha\\ Welington Matuzinho da Silva}
	\date{Dezembro 2022}
	\maketitle
	\thispagestyle{empty} % oculta numeraçao da pagina
	\newpage
	
	\setcounter{page}{1} % comeca contar
	\tableofcontents
	\newpage
	
	\section{\textbf{Introdução}}	
	

	
	\section{Descrição detalhada do projeto}
	
	\begin{itemize}
		\item 
		\item
	\end{itemize}

	\begin{enumerate}
		\item 
		\item
			\begin{enumerate}
				\item
				\item 
			\end{enumerate}
	\end{enumerate}

	\begin{description}
		\item[nuemrisjs]
	\end{description}

	\subsection{Conceitos:}
	
	\subsubsection*{Estatística descritiva}
	\newpage 
	\section{Concorrentes}
	
	Para realizar o planejamento do site solicitado, os desenvolvedores	precisaram fazer uma pesquisa sobre os produtos similares no mercado para conhecer sobre os possíveis principais concorrentes de sites educacionais para estatística. Foi realizada então, uma busca no site www.google.com.br com as seguintes palavras-chave “quiz estatístico”, “conteúdo estatístico” e “cálculo estatístico”. O retornado foi uma grande quantidade de resultados, porém nenhuma deles	correspondem as necessidades de um site para ensino de estatística para	ser usado por discentes da disciplina. Os principais resultados que podem ser concorrentes mesmo que inferiores do site aqui desenvolvido, são:\\

	\begin{enumerate}
		\item Racha Cuca.\\
		\\
	O Racha Cuca é um portal de entretenimento inteligente dedicado a todas as idades. Disponível em: https://rachacuca.com.br/. É possível encontrar no mesmo desde Jogos Online até Problemas de Lógica. Além disso, tem Palavras-Cruzadas, Caça Palavras, Anagramas, QuebraCabeças, Passatempos, Trivias e Quizzes enviados pelos usuários. Contam também com uma área de Educação , onde apresentam e disponibilizam explicações sobre assuntos do Ensino Médio. Embora seja um site educacional bem completo e apresente algo sobre estatística, esse não é o foco do mesmo.
	A figura abaixo mostra a tela inicial do site em questão:
		\item 
	\end{enumerate}
	
	
	 
\end{document}